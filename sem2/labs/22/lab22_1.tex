\documentclass[a4paper,12pt]{book}

\usepackage[T1]{fontenc}
\usepackage[utf8x]{inputenc}
\usepackage[russian]{babel}
\usepackage{geometry}

\geometry{left=2.5cm}
\geometry{right=1.5cm}
\geometry{top=0.7cm}
\geometry{bottom=2cm}

\begin{document}
\fontsize{14pt}{14pt}\selectfont
\parindent=0.0cm
{\small 302\ \ \ \ \ \ \ \ \ \ \ \ \ \ \ \ \ \ \ \ \ \ \ \ \ \ \ \ \ \ \ \ \ \ \ \ \ \ \ \ \ \ \ \ \ \ \ \ \ \ \ \ \ \ \ \ \ \ \ $\S$ 20. \textit{Частные производные. Дифференцируемость}
}\par
\vspace{1.5em}
\parindent=0.0cm
$-\infty <y<+\infty$, это означает, что функция $F(x, y)$ является постоянной, равной $f(x)$ на любой прямой, проходящей через точку $x$ интервала $(a, b)$ оси $Ox$ параллельно оси $Oy$. При этом\par
$$
\frac{\partial F(x, y)}{\partial x} = f'(x),\qquad \frac{\partial F(x, y)}{\partial y} = 0,\qquad dF(x, y) = df(x),
$$\par
$$
a<x<b,\qquad -\infty <y<+\infty .
$$\par
\parindent=0.7cm
Полезно для дальнейшего отметить в известном смысле обратный факт. Пусть $E\subset E^n$. Если функция $f^*(x_{1},...,x_{n},x_{n+1})$ определена на множестве\par
\parindent=0.0cm
$$
E^*=\lbrace f^*(x_{1},...,x_{n},x_{n+1}):(x_{1},...,x_{n})\in E, a<x_{n+1}<b\rbrace
$$\par
и\par
$$
\frac{\partial f^*(x_1,...,x_n,x_{n+1})}{\partial x_{n+1}}=0\mbox{ на } E^*, \eqno(20.38)
$$\par
то существует функция $f(x_1,...,x_n)$ от $n$ переменных, определенная на множестве $E$, и такая, что\par
$$
f^*(x_1,...,x_n,x_{n+1})=f(x_i,...,x_n)
$$\par
для всех\par
$$
(x_1,...,x_n) \in E, x_{n+1} \in (a, b).
$$\par
\parindent=0.7cm
В этом случае говорят, что \textit{функция $f^*$ фактически не зависит от переменной $x_{n+1}$}. В самом деле, из условия (20.38) следует, что функция $f^*$ постоянна как функция $x_{n+1}$ (см. лемму п. 11.2) при фиксированной точке $(x1,...,x_n) \in E$ и $x_{n+1} \in (a, b)$ имеем\par
$$
f^*(x_1,...,x_{n+1})=f(x_1,...,x_n,c).
$$\par
\parindent=0.0cm
Искомая функция $f$, очевидно, определяется равенством\par
$$
f(x_1,...,x_n)=f^*(x1,...,x_n,c),
$$\par
причем она не зависит от выбора $c \in (a, b)$.\par
\parindent=0.7cm
Из вышесказанного, в частности, следует, что формулы 1-3 для дифференциалов остаются справедливыми и в том случае, когда числа переменных, от которых зависят функции $u$ и $v$ - различны, так как всегда в силу указанного приема этот случай можно свести к вышеразобранному случаю одного числа переменных.\par
\vspace{1.5em}
\parindent=2.5cm
\textbf{20.5 Геометрический смысл частных производных}\par
\textbf{и полного дифференциала}\par
\vspace{1.0em}
Для большей геометрической наглядности и для того, чтобы не вводить новых понятий, в этом пункте ограничимся рассмотрением функций двух переменных.\par
\end{document}
