\documentclass[a4paper,12pt]{book}

\usepackage[T1]{fontenc}
\usepackage[T2A]{fontenc}
\usepackage[utf8x]{inputenc}
\usepackage[russian]{babel}
\usepackage{geometry}

\geometry{left=1.5cm}
\geometry{right=2.5cm}
\geometry{top=0.7cm}
\geometry{bottom=2cm}

\begin{document}
\fontsize{14pt}{14pt}\selectfont
\parindent=0.0cm
{\small 20.6. \textit{Производная по направлению}\ \ \ \ \ \ \ \ \ \ \ \ \ \ \ \ \ \ \ \ \ \ \ \ \ \ \ \ \ \ \ \ \ \ \ \ \ \ \ \ \ \ \ \ \ \ \ \ \ \ \ \ \ \ \ \ \ \ \ \ \ \ \ \ \ \ \ \ \ \ \ \ \ \ \ \ \ \ \ \ \ 307
}\par
\vspace{1.5em}
\parindent=0.0cm
но из (20.44) следует, что\par
$$
\frac{dx}{ds} = \cos \alpha,\quad \frac{dy}{ds} = \cos \beta,\quad \frac{dz}{ds} = \cos \gamma, \eqno{(20.45)}
$$\par
поэтому окончательно\par
$$
\frac{\partial f(M_0)}{\partial l} = \frac{\partial f(M_0)}{ds} = \frac{\partial f(M_0)}{\partial x} \cos \alpha + \frac{df(M_0)}{\partial y} \cos \beta + \frac{\partial f(M_0)}{\partial z} \cos \gamma. \eqno{(20.46)}
$$\par
Это и есть искомая формула.\par
\parindent=0.7cm
Таким образом, доказана следующая теорема.\par
\textbf{Теорема 7.} \textit{Пусть функция $f$ дифференцируема в точке $(x_0, y_0, z_0)$.}\par
\parindent=0.0cm
\textit{Тогда в этой точке функция $f$ имеет производные по любому направлению и эти производные находятся по формуле} (20.46).\par
\parindent=0.7cm
Очевидно, что по самому определению производной по направлению (от точки $M_0$ к точке $M_1$) функции точки она не зависит от выбора прямоугольной декартовой системы координат, а определяется только точками $M_0$ и $M_1$, или, что то же, точкой $M_0$ и вектором $\overrightarrow{M_0M_1}$ (кстати, этот факт сразу не виден из формулы (20.46)).\par
Вектор с координатами\par
$$
\frac{\partial f(M_0)}{\partial x},\quad \frac{\partial f(M_0)}{\partial y},\quad \frac{\partial f(M_0)}{\partial z}
$$\par
\parindent=0.0cm
называется \textit{градиентом} функции $f(M)$ в точке $M_0$ и обозначается $grad\ f$. Таким образом, если $\textbf{i}$, $\textbf{j}$ и $\textbf{k}$ - координатные орты, то\par
$$
grad\ f = \frac{\partial f}{\partial x} \textbf{i} + \frac{\partial f}{\partial y} \textbf{j} + \frac{\partial f}{\partial z} \textbf{k}. \eqno{(20.47)}
$$\par
\parindent=0.7cm
Часто оказывается удобным использование \textit{символ-вектора}\par
\parindent=0.0cm
\textit{Гамильтона$^{*)}$}.\par
$$
\bigtriangledown = \textbf{i} \frac{\partial}{\partial x} + \textbf{j} \frac{\partial}{\partial y} + \textbf{k} \frac{\partial}{\partial z},
$$\par
\parindent=0.0cm
называется \textit{наблой}. Набла является обозначением определенной операции, которую следует произвести над тем или иным объектом.\par
\parindent=0.7cm
Для функции $f$ по определению полагаем\par
$$
\bigtriangledown f = \textbf{i} \frac{\partial f}{\partial x} + \textbf{j} \frac{\partial f}{\partial y} + \textbf{k} \frac{\partial f}{\partial z}.
$$\par
Формально это равенство можно рассматривать как "произведение"\par
\parindent=0.0cm
вектора $\bigtriangledown$ на число $f$.\par
\parindent=0.7cm
Итак, $grad\ f$ и $\bigtriangledown f$ являются обозначениями одного и того же выражения.\par
\line(1,0){64}\par
\small *) У. Гамильтон (1805-1865) - английский математик.\par
\qquad \qquad \qquad \qquad \qquad \qquad \qquad \qquad \qquad \qquad \qquad \qquad 11*
\end{document}
